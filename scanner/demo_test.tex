\documentclass[11pt]{article}
\usepackage[pdftex]{graphicx}
\usepackage[utf8]{inputenc}
\usepackage[a4paper,left=1cm, right=1cm, top=1cm, bottom=1cm]{geometry}
\setlength{\parindent}{0pt}
\begin{document}
\noindent
Please mark the correct answer for the questions below.
The correct answer should be marked
by blurring the pattern on the right side of the corresponding letter.
If you want to skip the question, blur the pattern on the right side
of the question mark.\\
~\\
\includegraphics[height=0.7cm]{patterns/01.png}
The superposition theorem may be applied only: a)~To the linear circuits,
b)~To the circuits containing only passive elements,
c)~To the DC circuits,
d)~To the AC circuits

~\\

\includegraphics[height=0.7cm]{patterns/02.png}
The impedance of the capacitor: a)~does not depend on the frequency,
b)~is proportional to the frequency, c)~is inversely proportional to the
frequency, d) is always infinite

~\\

\includegraphics[height=0.7cm]{patterns/03.png}
The impedance of the inductance: a)~does not depend on the frequency,
b)~is proportional to the frequency, c)~is inversely proportional to the
frequency, d) is always infinite

~\\

\includegraphics[height=0.7cm]{patterns/04.png}
The impedance of the resistance: a)~does not depend on the frequency,
b)~is proportional to the frequency, c)~is inversely proportional to the
frequency, d) is always infinite

~\\

\includegraphics[height=0.7cm]{patterns/05.png}
Which of the below connections is not acceptable in the model of the circut:
  a)~parallel connection of two ideal current sources with different output currents,
  b)~series connection of two ideal voltage sources with different output voltages,
  c)~parallel connection of an ideal current source and an ideal voltage source,
  d)~series connection of two ideal current sources with different output currents,


~\\

\includegraphics[height=0.7cm]{patterns/06.png}
 Which from the listed amplifiers is {\bf not} suitable to amplify the DC component of the
 input signal:
 a)~Differential amplifier,
 b)~Common emitter or common source amplifier,
 c)~Noniverting amplifier with op-amp,
 d)~Inverting amplifier with op-amp

~\\

\includegraphics[height=0.7cm]{patterns/07.png}
 The load current is meassured as the voltage drop across the small resistor
 connected in series with the load. The accuracy of the metter is equal to 2\%, and
 the tolerance of the resistor is equal to 1.5\%. The accuracy of the measurement is
 equal to:
 a)~3\%, b)~2\%, c)~3.5\%, d)~1.5\%

~\\

\includegraphics[height=0.7cm]{patterns/08.png}
In order to measure the current in a circuit, an ammeter must:
 a)~be placed across the source;
 b)~be placed across the load;
 c)~be inserted into the circuit so the current flows through it;
 d)~all of these

~\\

\includegraphics[height=0.7cm]{patterns/09.png}
 In the bipolar NPN transistor, $\beta$=200A/A, $I_C$=1mA, $I_B$=10$\mu$A:
 a) transistor works in saturation state,
 b) transistor works in active state,
 c) this is impossible,
 d) transistor works in cutoff state,

~\\

\includegraphics[height=0.7cm]{patterns/10.png}
The real current source has internal resistance of 50$\Omega$. When shorted, it
  provides the current of 100mA. This source may be replaced with:
  a)~an ideal voltage source E=5V connected in parallel with resistance R=50$\Omega$,
  b)~an ideal voltage source E=5V connected in series with resistance R=50$\Omega$,
  c)~an ideal voltage source E=0.5V connected in parallel with resistance R=50$\Omega$,
  d)~an ideal voltage source E=0.5V connected in series with resistance R=50$\Omega$

\end{document}
